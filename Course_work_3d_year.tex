\documentclass[bachelor, och, coursework ]{SCWorks}
% параметр ---тип обучения ---одно из значений:
%    spec     ---специальность
%    bachelor ---бакалавриат (по умолчанию)
%    master   ---магистратура
% параметр ---форма обучения ---одно из значений:
%    och   ---очное (по умолчанию)
%    zaoch ---заочное
% параметр ---тип работы ---одно из значений:
%    referat    ---реферат
%    coursework ---курсовая работа (по умолчанию)
%    diploma    ---дипломная работа
%    pract      ---отчет по практике
%    pract      ---отчет о научно-исследовательской работе
%    autoref    ---автореферат выпускной работы
%    assignment ---задание на выпускную квалификационную работу
%    review     ---отзыв руководителя
%    critique   ---рецензия на выпускную работу
% параметр ---включение шрифта
%    times    ---включение шрифта Times New Roman (если установлен)
%               по умолчанию выключен
\usepackage{amsmath}
\usepackage[T2A]{fontenc}
\usepackage[utf8]{inputenc}
\usepackage{graphicx}
\usepackage[sort,compress]{cite}
\usepackage{amsmath}
\usepackage{amssymb}
\usepackage{amsthm}
\usepackage{fancyvrb}
\usepackage{longtable}
\usepackage{array}
\usepackage{caption}
\usepackage[english,russian]{babel}
\captionsetup[figure]{font= normalsize, labelfont=normalsize}
\DeclareUnicodeCharacter{00A0}{ }

\usepackage[colorlinks=true]{hyperref}
\usepackage{float}
\usepackage{caption}
\captionsetup[figure]{font= normalsize, labelfont=normalsize}


\begin{document}

% Кафедра (в родительном падеже)
\chair{Дискретной математики и информационных технологий}

% Тема работы
\title{Технологии формирования облака слов на платформе Python}

% Курс
\course{3}

% Группа
\group{322}

% Факультет (в родительном падеже) (по умолчанию "факультета КНиИТ")
%\department{факультета КНиИТ}

% Специальность/направление код ---наименование
%\napravlenie{02.03.02 "-----Фундаментальная информатика и информационные технологии}
%\napravlenie{02.03.01 "-----Математическое обеспечение и администрирование информационных систем}
\napravlenie{09.03.01 "---Информатика и вычислительная техника}
%\napravlenie{09.03.04 "-----Программная инженерия}
%\napravlenie{10.05.01 "-----Компьютерная безопасность}

% Для студентки. Для работы студента следующая команда не нужна.
%\studenttitle{Студентки}

% Фамилия, имя, отчество в родительном падеже
\author{Конорова Дмитрия Аркадьевича}

% Заведующий кафедрой
\chtitle{доцент, к.\,ф.-м.\,н.} % степень, звание
\chname{Л.\,Б.\,Тяпаев}

%Научный руководитель (для реферата преподаватель проверяющий работу)
\satitle{доцент, к.\,э.\,н.} %должность, степень, звание
\saname{Г.\,Ю.\,Чернышова}


% Семестр (только для практики, для остальных
% типов работ не используется)
\term{8}

% Наименование практики (только для практики, для остальных
% типов работ не используется)
%\practtype{}

% Продолжительность практики (количество недель) (только для практики,
% для остальных типов работ не используется)
\duration{4}

% Даты начала и окончания практики (только для практики, для остальных
% типов работ не используется)
\practStart{30.04.2021}
\practFinish{27.05.2021}

% Год выполнения отчета
\date{2023}

\maketitle

% Включение нумерации рисунков, формул и таблиц по разделам
% (по умолчанию ---нумерация сквозная)
% (допускается оба вида нумерации)
%\secNumbering


\tableofcontents


% Раздел "Обозначения и сокращения". Может отсутствовать в работе


% Раздел "Определения". Может отсутствовать в работе
%\definitions

% Раздел "Определения, обозначения и сокращения". Может отсутствовать в работе.
% Если присутствует, то заменяет собой разделы "Обозначения и сокращения" и "Определения"
%\defabbr


% Раздел "Введение"
\intro



\section{ Теоретическая часть}
\subsection{ Методы векторизации}
TF-IDF (term frequency - inverse document term frequency) --- это статистический показатель, применяемый для оценки важности слова в контексте категории, документа или коллекции документов. Он позволяет опредлить важность слова для данной коллекции документов. 
Term frequency (tf) выдает то, сколько раз слово t встречалось в документе d.
$tf(t, d) = f_{t, d}$.

Inverse document term frequency (IDF) используется для рассчета того, насколько редко встречается некоторое слово t среди всех документов. Слова, которые встречаются редко имеют большой IDF.
$idf(t, D) = log\frac{N}{|\{d\in D:t\in d|\}}$, где t - термин; D - множество документов; N - количество докуменов в корпусе; $|\{d\in D:t\in d\}|$ - число документов, в которых встречается термин t.


Методы векторизации:
{\begin{itemize}
    \item Bag of words;
    \item hot vectors;
    \item TF-IDF;
    \item word2vec;
    \item GLOVE;
    \item Fast text;
    \item CBOW.
\end{itemize}}
https://neptune.ai/blog/vectorization-techniques-in-nlp-guide


Ссылка на сайт пакета для R text2vec: https://cran.r-project.org/web/packages/text2vec/vignettes/text-vectorization.html#Text\_analysis\_pipeline

Примеры кода с использованием пакета text2vec:
\begin{verbatim}
    # define preprocessing function and tokenization function
prep_fun = tolower
tok_fun = word_tokenizer

it_train = itoken(train$review, 
             preprocessor = prep_fun, 
             tokenizer = tok_fun, 
             ids = train$id, 
             progressbar = FALSE)
vocab = create_vocabulary(it_train)


train_tokens = tok_fun(prep_fun(train$review))


#Аналогичное определение функции it_train
it_train = itoken(train_tokens, 
                  ids = train$id,
                  # turn off progressbar because it won't look nice in rmd
                  progressbar = FALSE)

vocab = create_vocabulary(it_train)
vocab





#N-GRAMS
t1 = Sys.time()
vocab = create_vocabulary(it_train, ngram = c(1L, 2L))
print(difftime(Sys.time(), t1, units = 'sec'))
## Time difference of 1.557114 secs
vocab = prune_vocabulary(vocab, term_count_min = 10, 
                         doc_proportion_max = 0.5)

bigram_vectorizer = vocab_vectorizer(vocab)

dtm_train = create_dtm(it_train, bigram_vectorizer)

t1 = Sys.time()
glmnet_classifier = cv.glmnet(x = dtm_train, y = train[['sentiment']], 
                 family = 'binomial', 
                 alpha = 1,
                 type.measure = "auc",
                 nfolds = NFOLDS,
                 thresh = 1e-3,
                 maxit = 1e3)
print(difftime(Sys.time(), t1, units = 'sec'))
## Time difference of 5.2114 secs




#NORMALIZAION
dtm_train_l1_norm = normalize(dtm_train, "l1")


#TF-IDF
vocab = create_vocabulary(it_train)
vectorizer = vocab_vectorizer(vocab)
dtm_train = create_dtm(it_train, vectorizer)

# define tfidf model
tfidf = TfIdf$new()
# fit model to train data and transform train data with fitted model
dtm_train_tfidf = fit_transform(dtm_train, tfidf)
# tfidf modified by fit_transform() call!
# apply pre-trained tf-idf transformation to test data
dtm_test_tfidf = create_dtm(it_test, vectorizer)
dtm_test_tfidf = transform(dtm_test_tfidf, tfidf)
\end{verbatim}

% Раздел "Заключение"
\conclusion



%Библиографический список, составленный вручную, без использования BibTeX
%
\begin{thebibliography}{99}
 \bibitem {1} Линник Л. А., Петросян М. М., Облако слов как метод компрессии информации научного текста // Наука. Информатизация. Технологии. Образование : материалы XIII международной научно-практической конференции, г. Екатеринбург, 24-28 февраля 2020 г. ---Екатеринбург : Издательство РГППУ, 2020. --- С. 99-108. 
\bibitem{2} Сапух Татьяна Викторовна Современные средства формирования лексических навыков учащихся на уроках английского языка (на примере облака слов) // АНИ: педагогика и психология. 2018. №3 (24).[Электронный ресурс]. URL: https://cyberleninka.ru/article/n/sovremennye-sredstva-formirovaniya-leksicheskih-navykov-uchaschihsya-na-urokah-angliyskogo-yazyka-na-primere-oblaka-slov (дата обращения: 03.09.2023).
\bibitem{3} Яркин П. А. Изучение экологического дискурса в контексте исследования экологического сознания // Экопсихологические исследования – 6: экология детства и психология устойчивого развития. 2020. №6.[Электронный ресурс]. URL: https://cyberleninka.ru/article/n/izuchenie-ekologicheskogo-diskursa-v-kontekste-issledovaniya-ekologicheskogo-soznaniya (дата обращения: 03.09.2023).
\bibitem{4} Journal of Teaching and Learning with Technology, Vol. 5, No. 1, July 2016, pp.16-32.
\bibitem{5} Davies, B.M., Mowforth, O.D., Khan, D.Z. et al. The development of lived experience-centered word clouds to support research uncertainty gathering in degenerative cervical myelopathy: results from an engagement process and protocol for their evaluation, via a nested randomized controlled trial. Trials 22, 415 (2021). [Электронный ресурс]. URL: https://doi.org/10.1186/s13063-021-05349-8 (дата обращения: 10.09.2023).
\bibitem{6} Umair, A., Masciari, E. Sentimental and spatial analysis of COVID-19 vaccines tweets. J Intell Inf Syst 60, 1–21 (2023). [Электронный ресурс]. URL: https://doi.org/10.1007/s10844-022-00699-4 (дата обращения: 25.09.2023).
\bibitem{7} Онлайн сервис формирования облака слов ``WordClouds.com''.  URL: https://www.wordclouds.com/ (дата обращения: 03.09.2023).
\bibitem{8} Онлайн сервис формирования облака слов ``wordcloud.online''. [Электронный ресурс]. URL: https://wordcloud.online/ (дата обращения: 04.09.2023).
\bibitem{9} Официальный сайт Python. [Электронный ресурс]. URL: https://www.python.org/
 \bibitem{10} Официальное сообщество VK ``Самарская область''. [Электронный ресурс]. URL: https://vk.com/samaroblast (дата обращения: 03.09.2023).
 \bibitem{11} Официальное сообщество VK ``Саратовская область''. [Электронный ресурс]. URL: https://vk.com/saratovoblgov (дата обращения: 03.09.2023).
 \bibitem{12} Официальное сообщество VK ``Республика Татарстан''. [Электронный ресурс]. URL: https://vk.com/nashtatarstan (дата обращения: 01.09.2023).
\bibitem{13} Список русских имён и фамилий. [Электронный ресурс]. URL: https://github.com/Raven-SL/ru-pnames-list/tree/master/lists (дата обращения: 20.09.2023).
\end{thebibliography}

%Библиографический список, составленный с помощью BibTeX
%
%\bibliographystyle{gost780uv}
%\bibliography{thesis}

% Окончание основного документа и начало приложений
% Каждая последующая секция документа будет являться приложением

\section{ПРИЛОЖЕНИЕ А}
\begin{verbatim}
\end{verbatim}
\end{document}
