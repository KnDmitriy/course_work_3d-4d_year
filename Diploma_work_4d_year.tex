\documentclass[bachelor, och, diploma ]{SCWorks}
% параметр ---тип обучения ---одно из значений:
%    spec     ---специальность
%    bachelor ---бакалавриат (по умолчанию)
%    master   ---магистратура
% параметр ---форма обучения ---одно из значений:
%    och   ---очное (по умолчанию)
%    zaoch ---заочное
% параметр ---тип работы ---одно из значений:
%    referat    ---реферат
%    coursework ---курсовая работа (по умолчанию)
%    diploma    ---дипломная работа
%    pract      ---отчет по практике
%    pract      ---отчет о научно-исследовательской работе
%    autoref    ---автореферат выпускной работы
%    assignment ---задание на выпускную квалификационную работу
%    review     ---отзыв руководителя
%    critique   ---рецензия на выпускную работу
% параметр ---включение шрифта
%    times    ---включение шрифта Times New Roman (если установлен)
%               по умолчанию выключен
\usepackage{amsmath}
\usepackage[T2A]{fontenc}
\usepackage[utf8]{inputenc}
\usepackage{graphicx}
\usepackage[sort,compress]{cite}
\usepackage{amsmath}
\usepackage{amssymb}
\usepackage{amsthm}
\usepackage{fancyvrb}
\usepackage{longtable}
\usepackage{array}
\usepackage{caption}
\usepackage[english,russian]{babel}
\captionsetup[figure]{font= normalsize, labelfont=normalsize}
\DeclareUnicodeCharacter{00A0}{ }

\usepackage[colorlinks=true]{hyperref}
\usepackage{float}
\usepackage{caption}
\captionsetup[figure]{font= normalsize, labelfont=normalsize}


\begin{document}

% Кафедра (в родительном падеже)
\chair{Дискретной математики и информационных технологий}

% Тема работы
\title{Разработка приложения для частотного анализа новостных сообщений}

% Курс
\course{4}

% Группа
\group{421}

% Факультет (в родительном падеже) (по умолчанию "факультета КНиИТ")
%\department{факультета КНиИТ}

% Специальность/направление код ---наименование
%\napravlenie{02.03.02 "-----Фундаментальная информатика и информационные технологии}
%\napravlenie{02.03.01 "-----Математическое обеспечение и администрирование информационных систем}
\napravlenie{09.03.01 "---Информатика и вычислительная техника}
%\napravlenie{09.03.04 "-----Программная инженерия}
%\napravlenie{10.05.01 "-----Компьютерная безопасность}

% Для студентки. Для работы студента следующая команда не нужна.
%\studenttitle{Студентки}

% Фамилия, имя, отчество в родительном падеже
\author{Конорова Дмитрия Аркадьевича}

% Заведующий кафедрой
\chtitle{доцент, к.\,ф.-м.\,н.} % степень, звание
\chname{Л.\,Б.\,Тяпаев}

%Научный руководитель (для реферата преподаватель проверяющий работу)
\satitle{доцент, к.\,э.\,н.} %должность, степень, звание
\saname{Г.\,Ю.\,Чернышова}


% Семестр (только для практики, для остальных
% типов работ не используется)
% \term{}

% Наименование практики (только для практики, для остальных
% типов работ не используется)
%\practtype{}

% Продолжительность практики (количество недель) (только для практики,
% для остальных типов работ не используется)
\duration{}

% Даты начала и окончания практики (только для практики, для остальных
% типов работ не используется)
\practStart{}
\practFinish{}

% Год выполнения отчета
\date{2025}




\maketitle

% Включение нумерации рисунков, формул и таблиц по разделам
% (по умолчанию ---нумерация сквозная)
% (допускается оба вида нумерации)
%\secNumbering


\tableofcontents


% Раздел "Обозначения и сокращения". Может отсутствовать в работе


% Раздел "Определения". Может отсутствовать в работе
%\definitions

% Раздел "Определения, обозначения и сокращения". Может отсутствовать в работе.
% Если присутствует, то заменяет собой разделы "Обозначения и сокращения" и "Определения"
%\defabbr


% Раздел "Введение"
\intro

Понятие "ключевое слово" уже давно получило точное определение.
Бойс и др. [7] назвали его !?суррогатом?!, который представляет тему
или содержание документа, что, в свою очередь, порождает другой
вопрос: что такое тема или содержание? Что в равной степени неуловимо. История стала свидетелем возникновения двух основных научных школ
: одной в области терминологии (TS) и другой в
области поиска информации (IR). Эти две
школы пересекались по мере продвижения в своих научных изысканиях. Терминологи, как правило, занимаются поиском терминов, которые являются
специфичными для конкретной технической области. Это полезно для систематизации
знаний, относящихся к этой области, в то время как люди, занимающиеся поиском информации, больше сосредоточены на идентификации терминов (которые
они называют терминами индексации), способных различать
документы для улучшения поиска документов.

\section{Методы извлечения ключевых слов}
\textbf{От 16.09.2024}
https://aclanthology.org/2021.hackashop-1.6/ - статья 2021 года. 
"Exploring Linguistically-Lightweight Keyword Extraction Techniques for
Indexing News Articles in a Multilingual Set-up".
В ней есть третий раздел "Methods", который можно полностью скопировать для работы и перевести. Особенность статьи - поиск и сравнение алгоритмов для 7 языков.

https://link.springer.com/article/10.1007/s42979-022-01481-7 - статья 2022 года. "Keyword Extraction: A Modern Perspective".
Эта статья намного подробнее предыдущей, в ней проводится более подробное рассмотрение алгоритмов и их более четкая классификация на: статистические, лингвистические, графовые и на алгоритмы с использованием машинного обучения. Проводится сравнение эффективности алгоритмов. Эта статья больше предыдущей. "Мы рассмотрим историю поиска ключевых
слов за последние 50 лет, отмечая различия и сходства между методами, появившимися за это время," - написано в этой статье.

https://www.mdpi.com/2076-3417/13/12/7228 - статья 2023 года. "Unlocking the Potential of Keyword Extraction: The Need for Access to High-Quality Datasets"
В этой статье рассматриваются и сравниваются алгоритмы KeyBERT, YAKE и RAKE. В двух предыдущих статья рассматриваемых алгоритмов намного больше (не меньше 8).

https://www.sciencedirect.com/science/article/abs/pii/S0957417422018607 - статья 2022 года. 
"Extracting keywords of educational texts using a novel mechanism based on linguistic approaches and evolutive graphs".
К ней нет бесплатного доступа.

https://link.springer.com/chapter/10.1007/978-981-99-6706-3_30 - статья 2023 года. 
"User Story-Based Automatic Keyword Extraction using Algorithms and Alalysis". 
Не имеет бесплатного доступа.





\textbf{От 03.02.2025}
\subsection{Back to the Basics: A Quantitative Analysis of Statistical and Graph-Based Term Weighting Schemes for Keyword Extraction}
Статья "Back to the Basics: A Quantitative Analysis of Statistical and
Graph-Based Term Weighting Schemes for Keyword Extraction",
https://aclanthology.org/2021.emnlp-main.638.pdf, статья ноября 2021 года.






% Раздел "Заключение"
\conclusion







%Библиографический список, составленный вручную, без использования BibTeX
%
\begin{thebibliography}{99}
 \bibitem {1} Линник Л. А., Петросян М. М., Облако слов как метод компрессии информации научного текста // Наука. Информатизация. Технологии. Образование : материалы XIII международной научно-практической конференции, г. Екатеринбург, 24-28 февраля 2020 г. ---Екатеринбург : Издательство РГППУ, 2020. --- С. 99-108. 
\bibitem{2} Сапух Татьяна Викторовна Современные средства формирования лексических навыков учащихся на уроках английского языка (на примере облака слов) // АНИ: педагогика и психология. 2018. №3 (24).[Электронный ресурс]. URL: https://cyberleninka.ru/article/n/sovremennye-sredstva-formirovaniya-leksicheskih-navykov-uchaschihsya-na-urokah-angliyskogo-yazyka-na-primere-oblaka-slov (дата обращения: 10.05.2024).
\bibitem{3} Яркин П. А. Изучение экологического дискурса в контексте исследования экологического сознания // Экопсихологические исследования – 6: экология детства и психология устойчивого развития. 2020. №6.[Электронный ресурс]. URL: https://cyberleninka.ru/article/n/izuchenie-ekologicheskogo-diskursa-v-kontekste-issledovaniya-ekologicheskogo-soznaniya (дата обращения: 20.04.2024).
\bibitem{4} Journal of Teaching and Learning with Technology, Vol. 5, No. 1, July 2016, pp.16-32.
\bibitem{5} Umair, A., Masciari, E. Sentimental and spatial analysis of COVID-19 vaccines tweets. J Intell Inf Syst 60, 1–21 (2023). [Электронный ресурс]. URL: https://doi.org/10.1007/s10844-022-00699-4 (дата обращения: 10.05.2024).
\bibitem{6} Liberatore F., Camacho-Collados J. Back to the Basics: A Quantitative Analysis of Statistical and Graph-Based Term Weighting Schemes for Keyword Extraction. 2021. In Proceedings of the 2021 Conference on Empirical Methods in Natural Language Processing, 8089–8103 pp. Online and Punta Cana, Dominican Republic. Association for Computational Linguistics.

\bibitem{7}Gumińska, Urszula \& Poniszewska-Maranda, Aneta \& Ochelska-Mierzejewska, Joanna. (2022). Systematic Comparison of Vectorization Methods in Classification Context. Applied Sciences. 12. 5119. 10.3390/app12105119.  

\bibitem{8}Официальный сайт CRAN для скачивания R для macOS. [Электронный ресурс]. URL:   https://cran.r-project.org/bin/macosx/ (дата обращения: 08.05.2024).







\bibitem{9}Документация библиотеки udpipe. [Электронный ресурс]. URL: https://cran.r-project.org/web/packages/udpipe/vignettes/udpipe-annotation.html (дата обращения: 10.05.2024).



 \bibitem{10} Список русских имён и фамилий. [Электронный ресурс]. URL: https://github.com/Raven-SL/ru-pnames-list/tree/master/lists (дата обращения: 15.04.2024).


\bibitem{11} Официальный сайт shiny. [Электронный ресурс]. URL: https://shiny.posit.co/ (дата обращения: 09.05.2024).

\end{thebibliography}

%Библиографический список, составленный с помощью BibTeX
%
%\bibliographystyle{gost780uv}
%\bibliography{thesis}

% Окончание основного документа и начало приложений
% Каждая последующая секция документа будет являться приложением

\begin{center}
    \section*{\makebox[\linewidth]{ПРИЛОЖЕНИЕ А}}
\end{center}
%\section*{ ПРИЛОЖЕНИЕ А}
\addcontentsline{toc}{section}{ПРИЛОЖЕНИЕ А}
\begin{verbatim}
install_or_load_pack <- function(pack){
  create.pkg <- pack[!(pack %in% installed.packages()[, 
  "Package"])]
  if (length(create.pkg))
    install.packages(create.pkg, dependencies = TRUE)
  sapply(pack, require, character.only = TRUE)
}
packages <- c("ggplot2",  "data.table", "wordcloud", "tm", 
"wordcloud2", "tidytext", "devtools", 
"dplyr", 'tidyverse', 'readxl', 
'udpipe', 'writexl', 'xlsx', 'rlang')
install_or_load_pack(packages)

library(shiny)
library(readxl)
library(dplyr)
library(plyr)
library(tidytext)
library(ggplot2)
library(wordcloud2)
library(lsa)


wordcloud2a <- function (data, size = 1, minSize = 0, 
gridSize = 0,
fontFamily = "Segoe UI",               
     fontWeight = "bold", color = "random-dark", 
     backgroundColor = "white", 
     minRotation = -pi/4, maxRotation = pi/4, shuffle = TRUE, 
     rotateRatio = 0.4, shape = "circle", ellipticity = 0.65, 
     widgetsize = NULL, figPath = NULL, hoverFunction = NULL) 
{
  if ("table" %in% class(data)) {
    dataOut = data.frame(name = names(data), freq = 
    as.vector(data))
  }
  else {
    data = as.data.frame(data)
    dataOut = data[, 1:2]
    names(dataOut) = c("name", "freq")
  }
  if (!is.null(figPath)) {
    if (!file.exists(figPath)) {
      stop("cannot find fig in the figPath")
    }
    spPath = strsplit(figPath, "\\.")[[1]]
    len = length(spPath)
    figClass = spPath[len]
    if (!figClass %in% c("jpeg", "jpg", "png", "bmp", "gif")) {
      stop("file should be a jpeg, jpg, png, bmp or gif file!")
    }
    base64 = base64enc::base64encode(figPath)
    base64 = paste0("data:image/", figClass, ";base64,", 
                    base64)
  }
  else {
    base64 = NULL
  }
  weightFactor = size * 180/max(dataOut$freq)
  settings <- list(word = dataOut$name, freq = dataOut$freq, 
                   fontFamily = fontFamily, 
                   fontWeight = fontWeight, 
                   color = color, 
                   minSize = minSize, weightFactor = weightFactor, 
                   backgroundColor = backgroundColor, 
                   gridSize = gridSize, minRotation = minRotation, 
                   maxRotation = maxRotation, 
                   shuffle = shuffle, rotateRatio = rotateRatio, 
                   shape = shape, 
                   ellipticity = ellipticity, figBase64 = base64, 
                   hover = htmlwidgets::JS(hoverFunction))
  chart = htmlwidgets::createWidget("wordcloud2", settings, 
        width = widgetsize[1], 
        height = widgetsize[2], 
        sizingPolicy = htmlwidgets::sizingPolicy(viewer.padding 
        = 0, 
     browser.padding = 0, browser.fill = TRUE))
  chart
}




load_stopwords <- function() {
  female_names_rus <- read.csv("female_names_rus.txt",
  header=FALSE)
  male_names_rus <- read.csv("male_names_rus.txt", header=FALSE)
  male_surnames_rus <- read.csv("male_surnames_rus.txt",
  header=FALSE)
  extra_stop_words <- c('и','димитровграда', 'димитровград',
  'ульяновскаяобласть', 'ульяновск', 'ульяновский', 'саранск', 
  'саранска', 'мордовие', 'рм', 'рма', 'мордовия', 'мордовский', 
  'заец', 'idюрий', 'главамарийэл', 'марий', 'эл', 'марийэл', 
  'эть', 'васил', 'чурин', 'кировский', 'кировскаяобласть',
  'вятский', 'мельниченко', 'месяц', 'оренбургнуть', 
  'объясняемрф', 'провести',  'инвестор', 'вести', 'реализация', 
  'башкортостанный', 'радий', 'подписать', 'проект', 'пермский', 
  'пермскийкрай', 'край', 'прикамья', 'краевой', 'задача', 
  'важно', 'оренбуржец', 'оренбург',  'новость', 'подчеркнуть', 
  'оренбуржье', 'оренбургский', 
  'оренбургскаяобласть', 'поддержка', 'часть', 'км', 
  'валерийрадаеть', 'олегнуть', 'должен', 'около', 'рассказать', 
  'глава', 'губернатор', 'развитие', 'январь', 'февраль', 'март', 
  'апрель', 'май', 'июнь', 'июль', 'август', 'сентябрь', 
  'октябрь', ноябрь', 'декабрь', 'город', 'ecom', 'казань', 
  'подробность', 'подробный', 'радия', 'процент', 
  'уф', 'часть', 'вопрос',  'делать',  'сделать',
  'благодаря', 'участие', 'пройти', 'идти', 'создать', 
  'создавать', 'дать',  'рамка', 'место', 'первый', 'получить', 
  'удмуртия', 'радай', 'юлие', 'пензенский', 'пенза', 
  'пензенскаяобласть', 'новый', 'лучший', 'самый', 'работа', 
  'рабочий', 'работать', 'региональный', 'нижегородскаяобладать', 
  'clubнижегородский', 'нижегородскаяобласть', 'нижегородский', 
  'нижний', 'новгород', 'чувашия', 'чувашие', 'обть', 
  "бaшҡортостать", "бaшҡортостан", 'командахабиров', 'рб', 
  'миллиард', 'башкирия', 'башкортостан', 'башкортостана', 
  'мый', 'аный', 'мухаметшина', 'мухаметшин', 'реть', 
  'рф', 'день', 'отметить', 'число', 'миллион', 'ход', 
  'президент','страна',  'тысяча', 'рубль', 'доллар', 
  'район', 'итог', 'татарстан', 
  'татарстать', 'российский', 'ма', 'область', 'республика', 
  'саратовский', 'татарстан', 'татарстана', 
  'самарский','экономический', 'экономика', 'регион', 'год', 
  "миннихан", "рт", "россия", "рустам", "руст", 'россия', 
  'конкурентоспособность', 'инновация', 'инвестиция', 
  'инвестиционный', 'рустамминнихан', 'дмитрий', 'азаров', 
  'саратовскаяобласть', 'саратовская', 'самарскаяобласть', 
  'азар', 'стать', 'rn«', 'твой', 'сих', 'ком', 'свой',
    'слишком', 'нами', 'всему', 'будь', 'саму', 
    'чаще', 'ваше', 'наш', 'затем', 'еще', 
    'наши', 'ту', 'каждый',
    'мочь', 'весь', 'этим', 'наша', 'своих', 
    'оба', 'который', 'зато', 'те', 'вся', 'ваш', 
    'такая', 'теми', 'ею', 'нередко',
    'также', 'чему', 'собой', 'нем', 'вами', 
    'ими', 'откуда', 'такие', 'тому', 'та', 
    'очень', 'нему',  'д', 'алло', 'оно', 'кому', 
    'тобой', 'таки', 'мой', 'нею', 'ваши', 
    'ваша', 'кем', 'мои', 'однако', 'сразу', 'свое', 
    'ними', 'всё', 'неё', 'тех', 'хотя', 
    'всем', 'тобою', 'тебе', 'одной', 'другие',
    'буду', 'моё', 'своей', 'такое', 'всею', 'будут', 
    'своего',  'кого', 'свои', 'мог', 'нам', 'особенно', 
    'её','наше', 'кроме', 'вообще', 'вон', 'мною', 'никто', 
    'это', 'изза', 'именно', 'поэтому', 'будьт', 'являться', 
    'чувашский', 'тыса', 'смочь', 'ваший', 'гльба', 
    'ать', 'уть', 'ивать', 'ольги', 'пенз', 'ер', 'иметь', 
    'олегнуть', 'сг', 'например', 'сообщить', 
    'сообщать', 'среди', 'нть', 'пер', 'зспермь', 'края', 
    'ради',  'назвать', 'важный')
  stopwords_combined <- paste(c(stopwords("russian"), 
  extra_stop_words,
                                tolower(male_names_rus$V1),
                                tolower(male_surnames_rus$V1),
                                tolower(female_names_rus$V1)),
                                collapse = "|")
}

stopwords_combined <- load_stopwords()

ui <- fluidPage(
  titlePanel("Анализ регионов по разным периодам"),
  tabsetPanel(id = "tabs",
              tabPanel("Период 1",
                       sidebarLayout(
                         sidebarPanel(
                           fileInput("file1",
                           "Выберите Excel файл 1", accept = 
                           ".xlsx"),
                           actionButton("analyze1",
                           "Анализировать файл 1"),
                           
                         ),
                         mainPanel(
                           plotOutput("barPlot1"),
                           wordcloud2Output("wordcloud1"),
                           tableOutput("wordTable1")
                         )
                       )
              ),
              tabPanel("Период 2",
                       sidebarLayout(
                         sidebarPanel(
                           fileInput("file2",
                           "Выберите Excel файл 2", accept = 
                           ".xlsx"),
                           actionButton("analyze2", 
                           "Анализировать файл 2"),
                          
                         ),
                         mainPanel(
                           plotOutput("barPlot2"),
                           wordcloud2Output("wordcloud2"),
                           tableOutput("wordTable2")
                         )
                       )
              ),
              tabPanel("Период 3",
                       sidebarLayout(
                         sidebarPanel(
                           fileInput("file3", 
                           "Выберите Excel файл 3", accept = 
                           ".xlsx"),
                           actionButton("analyze3",
                           "Анализировать файл 3"),
                           
                         ),
                         mainPanel(
                           plotOutput("barPlot3"),
                           wordcloud2Output("wordcloud3"),
                           tableOutput("wordTable3")
                         )
                       )
              ),
              tabPanel("Сравнить файлы",
                       
                       actionButton("compare_files_btn", 
                       "Сравнить введенные файлы"),
                       tableOutput("compare_files_table")
              )
  )
)

server <- function(input, output, session) {
  files_preprocessed_data <- reactiveValues()
  clean_corpus <- function(corpus_to_use){
    corpus_to_use %>%
      tm_map(removePunctuation) %>%
      tm_map(stripWhitespace) %>%
      tm_map(content_transformer(function(x) 
      iconv(x, to='UTF-8'))) %>%
      tm_map(removeNumbers) %>%
      tm_map(content_transformer(tolower)) 
  }
  get_preprocessed_texts_word_list <- function(file) {
    req(file)
    input_data <- read_excel(file$datapath)
    load_stopwords()
    corp_city_df <- clean_corpus(VCorpus(VectorSource(
    input_data)))
    corp_city_df[["1"]][["content"]] <- gsub("[
    \U{1F600}-\U{1F64F}
    \U{1F300}-\U{1F5FF}\U{1F680}-\U{1F6FF}\U{1F1E0}-\U{1F1FF}
    \U{2500}-\U{2BEF}\U{2702}-\U{27B0}\U{24C2}-\U{1F251}
    \U{1f926}-\U{1f937}\U{10000}-\U{10ffff}\u{2640}-\u{2642}
    \u{2600}-\u{2B55}\u{200d}\u{23cf}\u{23e9}\u{231a}\u{fe0f}
    \u{3030}]","", corp_city_df[["1"]][["content"]], 
    perl = TRUE)
    if (!file.exists('russian-gsd-ud-2.5-191206.udpipe'))
    {
      gsd_model_raw <- udpipe_download_model(language = 
      "russian-gsd")
    }
    gsd_model <- udpipe_load_model(file =
      'russian-gsd-ud-2.5-191206.udpipe')
    x <- udpipe_annotate(gsd_model, x = corp_city_df[["1"]]
      [["content"]],  parser = "none")
    x <- as.data.frame(x)
    x$lemma <- noquote(x$lemma)
    x$lemma <- str_replace_all(x$lemma, "[[:punct:]]", "")
    tmp <- x$lemma
    tmp <- str_replace_all(x$lemma, paste("\\b(", 
    stopwords_combined,
    ")\\b"), "")
    tmp <- str_replace_all(tmp, '№', '')
    tmp <- str_replace_all(tmp, '−', '')
    tmp <- str_replace_all(tmp, '—', '')
    tmp <- str_replace_all(tmp, 'правительстворазвитие', 
    'правительство')
    tmp <- str_replace_all(tmp, 'правительстворб', 
      'правительство')
    tmp <- str_replace_all(tmp, 'цифровый', 'цифровой')
    tmp <- str_replace_all(tmp, 'научныймощность', 
    'научный мощность')
    tmp <- str_replace_all(tmp, 'club', '')
    tmp <- str_replace_all(tmp, 'правительствомарийэть', 
      'правительство')
    tmp <- str_replace_all(tmp, 'цура', 'цур')
    tmp <- tmp[sapply(tmp, nchar) > 0]
    return(tmp)
  }
  analyze_and_render <- function(file_input, 
  plot_output, table_output, wordcloud_output) {
    preprocessed_texts_word_list <- 
      get_preprocessed_texts_word_list(file_input)
    d <- as.data.frame(sort(table(
    preprocessed_texts_word_list), decreasing = TRUE))
    colnames(d) <- c("word", "freq")
    word_freq <- d
    d$tf <- d$freq / nrow(d)
    output[[plot_output]] <- renderPlot({
      ggplot(word_freq[1:10, ], 
      aes(x = reorder(word, freq), y = freq)) +
        geom_bar(stat = "identity") +
        coord_flip() +
        labs(title = "Наиболее часто встречающиеся слова", 
          x = "Слова", y = "Частота встречаемости") +
        theme_gray(base_size = 18)
    })
    output[[table_output]] <- renderTable({
      colnames(word_freq) <- c("Слово", 
        "Частота встречаемости слова в корпусе текстов")
      head(word_freq, 10)
    })
    output[[wordcloud_output]] <- renderWordcloud2({
      wordcloud2a(word_freq, size = 0.45)
    })
    return(d)
  }
  observeEvent(input$analyze1, {
    files_preprocessed_data[["df_1"]] <- 
      analyze_and_render(input[["file1"]],
      "barPlot1", "wordTable1", "wordcloud1")
  })
  observeEvent(input$analyze2, {
    files_preprocessed_data[["df_2"]] <- 
      analyze_and_render(input[["file2"]], 
      "barPlot2", "wordTable2", "wordcloud2")
  })
  observeEvent(input$analyze3, {
    files_preprocessed_data[["df_3"]] <- 
    analyze_and_render(input[["file3"]], 
    "barPlot3", "wordTable3", "wordcloud3")
  })
  observeEvent(input[["compare_files_btn"]], {
    d_all <- Filter(Negate(is.null), 
    list(files_preprocessed_data[["df_1"]], 
    files_preprocessed_data[["df_2"]], 
    files_preprocessed_data[["df_3"]])) 
    cos.mat <- NULL
    if (length(d_all) == 1) {
    }
    if (length(d_all) == 2) {
      d_all <- full_join(d_all[[1]], d_all[[2]], by='word')
      d_all <- d_all %>% replace(is.na (.), 0)
      tf_idf <- select(d_all, 'word', 'freq.x', 'tf.x',
      'freq.y','tf.y')
      names(tf_idf) <- c('word', 'freq1', 'tf1', 'freq2', 
      'tf2')
      tdm_df <- select(d_all, 'word', 'freq.x', 'freq.y')
      names(tdm_df) <- c('word', 'freq1', 'freq2')
      tdm_df <- tdm_df %>% mutate(num_of_occurrences = 
      rowSums(select(tdm_df, 'freq1', 'freq2') != 0))
      tdm_df <- tdm_df %>% mutate(idf = log(4 / 
      (1 + num_of_occurrences) + 1))
      tf_idf <- tf_idf %>% mutate(num_of_occurrences = 
      tdm_df$num_of_occurrences)
      tf_idf <- tf_idf %>% mutate(idf = tdm_df$idf)
      tf_idf <- tf_idf %>% mutate(tf_idf1 = tf1 * idf)
      tf_idf <- tf_idf %>% mutate(tf_idf2 = tf2 * idf)
      tf_idf_only <- select(tf_idf, 'tf_idf1', 'tf_idf2')
      names(tf_idf_only) <- c("Период 1", "Период 2")
      cos.mat <- cosine(as.matrix(tf_idf_only))  
    }
    if (length(d_all) == 3) {
      d_all <- full_join(full_join(d_all[[1]], 
      d_all[[2]], by='word'), d3, by='word')
      d_all <- d_all %>% replace(is.na (.), 0)
      tf_idf <- select(d_all, 'word', 'freq.x', 'tf.x', 
      'freq.y','tf.y', 'freq', 'tf')
      names(tf_idf) <- c('word', 'freq1', 'tf1', 'freq2',
      'tf2', 'freq3', 'tf3')
      tdm_df <- select(d_all, 'word', 'freq.x', 'freq.y',
      'freq')
      names(tdm_df) <- c('word', 'freq1', 'freq2', 'freq3')
      tdm_df <- tdm_df %>% mutate(num_of_occurrences = 
      rowSums(select(tdm_df, 'freq1', 'freq2', 'freq3') 
      != 0))
      tdm_df <- tdm_df %>% mutate(idf = log(4 / 
      (1 + num_of_occurrences) + 1))
      tf_idf <- tf_idf %>% mutate(num_of_occurrences = 
      tdm_df$num_of_occurrences)
      tf_idf <- tf_idf %>% mutate(idf = tdm_df$idf)
      tf_idf <- tf_idf %>% mutate(tf_idf1 = tf1 * idf)
      tf_idf <- tf_idf %>% mutate(tf_idf2 = tf2 * idf)
      tf_idf <- tf_idf %>% mutate(tf_idf3 = tf3 * idf)
      tf_idf_only <- select(tf_idf, 'tf_idf1', 
      'tf_idf2', 'tf_idf3')
      names(tf_idf_only) <- c("Период 1", "Период 2",
      "Период 3")
      cos.mat <- cosine(as.matrix(tf_idf_only))
    }
    output$compare_files_table <- renderTable({
      cos.mat
    })
  })
}

shinyApp(ui = ui, server = server)

\end{verbatim}
\end{document}
